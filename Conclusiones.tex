\chapter{Conclusiones}

\section{Conclusiones del trabajo realizado}
$\bullet$ A lo largo de este proyecto la concentración y el esfuerzo fueron dedicados al perfeccionamiento de un \textbf{biosensor} impreso en carbono por serigrafía mediante tinta de nanopartículas de oro impresa por el método \textit{Inkjet}. A su vez se buscó lograr una fabricación escalable para, en caso que sea viable, poder industrializar la fabricación.


$\bullet$ Se concluye que la tinta posee un anclaje aceptable con la tinta de carbono, logrando realizarse diseños con una precisión aceptable. Sin embargo, sobre un sustrato rugoso \textit{Valox} no logra un anclaje suficiente para poder mantener el diseño deseado.

$\bullet$ Para un correcto curado de cada capa de tinta, se debe curar por 80 minutos en \textit{Hot Plate} a 80ºC.

$\bullet$ Dos capas de tinta con nanopartículas de oro posee un espesor 3 veces menor a la de una impresión de carbono por serigrafía.

$\bullet$ La resistividad de dos capas curadas es aproximadamente 5 veces mayor a la del oro puro.

$\bullet$ La capacidad electroquímica de un \emph{WE} de carbono con 2 capas de nanopartículas de oro curadas es comparable a la de un \emph{WE} de oro depositado por \textit{Sputtering}. Recordando la ecuación de Randles-Sevcik \ref{ecuacion3}, la diferencia puede atribuirse, en parte, a la diferencia en las áreas efectivas de los \emph{WE}. Si bien geométricamente son idénticos (1 mm de diámetro), la rugosidad genera una diferencia en el valor efectivo usado en los cálculos. Las ventajas de la impresión por \textit{inkjet} son la facilidad de fabricación y el costo considerablemente menor. 

$\bullet$ La impresión de dos capas de tinta de nanopartículas de oro por \textit{Inkjet} sobre un sustrato liso a nivel macroscópico como el PET, se acerca a la curva de voltametría cíclica del oro depositado por \textit{sputtering} sobre silicio, una superficie lisa a nivel nanométrico.

$\bullet$ Se logró la correcta eyección de gotas y su definición del espaciado entre las mismas para la tinta dieléctrica SU-8. Esto permitió realizar las impresiones con el diseño deseado y motivó a continuar con el desarrollo de esta tinta para futuros proyectos.

\section{Trabajo futuro}
El presente proyecto sobre biosensores impresos por método \textit{Inkjet} promueve, al menos, dos pasos inmediatos a desarrollar. El primero vinculado a la composición e impresión de la tinta de carbono, para independizarse del paso intermedio de serigrafía. El segundo, y con mayor impacto, lograr la composición de una tinta dieléctrica y su impresión por método \textit{Inkjet} para la formación de microcubetas que contendrán la muestra a analizar sobre el electrodo, sin necesidad de pasar por otros procesos de fabricación.

En el apartado de Desarrollo Experimental (Capítulo 3, \hyperref[sec:tinta_dielec]{apartado 3.5}) se menciona la formulación de la tinta dieléctrica y las primeras impresiones de prueba con la misma, las cuales deberían tenerse en cuenta para la continuidad del desarrollo de la tecnología y las próximas mejoras en los biosensores.

\nocite{Banica}
\nocite{Prudenziati}
\nocite{Voros}
\nocite{Poc}
\nocite{PosterPoc1}
\nocite{DMPDatasheet}
\nocite{AgParticlesDimatix1}
\nocite{AgParticlesDimatix2}