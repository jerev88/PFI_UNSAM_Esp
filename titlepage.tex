%% ## Construye tu propia portada ##
%% 
%% Una portada se conforma por una secuencia de "Blocks" que incluyen
%% piezas individuales de informaci'on. Un "Block" puede incluir, por
%% ejemplo, el t'itulo del documento, una im'agen (logotipo de la universidad),
%% el nombre del autor, nombre del supervisor, u cualquier otra pieza de
%% informaci'on.
%%
%% Cada "Block" aparece centrado horizontalmente en la p'agina y,
%% verticalmente, todos los "Blocks" se distruyen de manera uniforme 
%% a lo largo de p'agina.
%%
%% Nota tambi'en que, dentro de un mismo "Block" se pueden cortar
%% lineas usando el comando \\
%%
%% El tama'no del texto dentro de un "Block" se puede modificar usando uno de
%% los comandos:
%%   \small      \LARGE
%%   \large      \huge
%%   \Large      \Huge
%%
%% Y el tipo de letra se puede modificar usando:
%%   \bfseries - negritas
%%   \itshape  - it'alicas
%%   \scshape  - small caps
%%   \slshape  - slanted
%%   \sffamily - sans serif


\begin{titlepage}
  \TitleBlock[\bigskip]{\scshape\insertinstitution}
  \TitleBlock[\bigskip]{\scshape\insertfaculty}
  \TitleBlock[\bigskip]{\includegraphics[height=4cm]{Figuras/logo_unsam}}
  \TitleBlock[\bigskip]{\bfseries\Large\scshape\inserttitle}
  \TitleBlock[\bigskip]{\scshape\large
  Proyecto final integrador para obtener el grado de \insertdegree}
  \TitleBlock{\bfseries\scshape Autor}  
  \TitleBlock[\vspace*{0,1cm}]{\scshape\insertauthor}
  \TitleBlock{jeremias.vozzi$@$gmail.com}  
  \TitleBlock[\bigskip]{\bfseries\scshape Tutores}
  \TitleBlock{\small\insertsupervisor}
  \TitleBlock[\vspace*{1cm}]{\bfseries\large\insertsubmitdate}
\end{titlepage}


%% Nota 2:
%% Normalmente, el espacio entre "Blocks" se extiende de modo que el
%% contenido se reparte uniformemente sobre toda la p'agina. Este
%% comportamiento se puede modificar para mantener fijo, por ejemplo, el
%% espacio entre un par de "Blocks". Escribiendo:
%%   \TitleBlock{Bloque 1}
%%   \TitleBlock[\bigskip]{Bloque2}
%% se deja un espacio "grande" y de tama~no fijo entre el bloque 1 y 2.
%% Adem'as de \bigskip est'an tambi'en \smallskip y \medskip. Si necesitas
%% aun m'as control puedes usar tambi'en, por ejemplo, \vspace*{2cm}[\vspace*{0,5cm}].


