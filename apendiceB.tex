\chapter{Proceso de cierre del proyecto}\label{chap:apendiceB}

\section{Entregables y espectativas}
Los entregables del proyecto han sido aceptados por los interesados, obteniendo a través de las caracterizaciones realizadas los resultados esperados. Estos incluyen la correcta puesta a punto de la impresora Fujifilm Dimatix DMP2850 para las tintas utilizadas, los diseños a utilizar para los entregables, la fabricación del producto y sus caracterizaciones para obtener las especificaciones finales.

Las espectativas han sido cumplidas y se han superado, pudiendo agregar al proyecto las pruebas con tinta dielectrica, propuesto como trabajo futuro. Se cumplió con los tiempos establecidos, la cantidad y características de los entregables propuestos y no se precisó utilizar más materiales de los proyectados desde el comienzo.

\section{Lecciones aprendidas}
Es crítico ser realista a la hora de estimar los tiempos de ejecución de la tareas de un proyecto.

Realizar un exhaustivo estudio sobre las teorías y otros trabajos realizados aplicados al proyecto ayuda a tener una mejor visión sobre la planificación del mismo y, a su vez, mejora el desarrollo de las tareas.

En un proyecto de investigación es vital tomar nota sobre todos los trabajos desarrollados y resultados obtenidos. Tener códigos de seguimiento de los distintos entregables simplifica el desarrollo y el conocimiento del historial de cada uno.

Mantener los archivos de seguimiento del proyecto actualizados día a día ayuda a mejorar la cantidad y calidad de la información que se vuelca.

Con respecto al proyecto realizado se enumeran los siguientes puntos a tener en cuenta para futuros trabajos:

$\bullet$ Antes de iniciar la impresora, verificar que el área de impresión este libre de obstáculos.

$\bullet$ Mantener los cartuchos en posición vertical, con el cabezal hacia abajo, para evitar la entrada de aire al canal del reservorio.

$\bullet$ Si se fija el sustrato a la platina con cinta adhesiva, se deberá tener en cuenta el espesor de la misma para sumarlo en la configuración de la impresión.

$\bullet$ Antes de hacer una impresión verificar que los eyectores estén funcionando correctamente y las gotas de tinta tengan un vuelo alineado.

$\bullet$ Controlar el estado del pad de limpieza periodicamente, si el cabezal comienza a llenarse de tinta sobre su base puede deberse a la saturación del material secante.

